\documentclass[11pt,a4paper, final, factor=1100, stretch=18, shrink=18]{moderncv}

\usepackage{color}
\usepackage{fontspec}

% Stop bugging me about \es
\usepackage{xspace}
% moderncv themes
\moderncvstyle{classic}                             % style options are 'casual' (default), 'classic', 'banking', 'oldstyle' and 'fancy'
%\moderncvcolor{blue}                               % color options 'black', 'blue' (default), 'burgundy', 'green', 'grey', 'orange', 'purple' and 'red'

% the (custom) color which will be used in the cv
\definecolor{color1}{RGB}{50, 50, 93}

% scale the page layout (depreciated for more complicated stuff)
%\usepackage[scale=0.75]{geometry}

% change width of the column with the dates
\setlength{\hintscolumnwidth}{2.5cm} 

%\PassOptionsToPackage[final, factor=1100, stretch=18, shrink=18 ]{microtype} 	
% The final option overrides global defaults. It greatly improves general appearance of the text. The stretch and shrink reduce bluriness[20,20 default]. The factor increases protrusion amount by 10%, [default 1000]
% Tracking allows for small caps, like in cites to be adjusted. The activate commands are to set protrusion.

\microtypecontext{spacing=nonfrench} 						% To preserve interword spacing via \nonfrenchspacing.
 
\SetExtraKerning[unit=space] 								% These produce more effects, from microtype with kerning.
    {encoding={*}, family={bch}, series={*}, size={footnotesize,small,normalsize}}
    {\textendash={400,400}, 								% en-dash, add more space around it
     "28={ ,150}, 											% Left bracket, add space from right
     "29={150, },											% Right bracket, add space from left
     \textquotedblleft={ ,150}, 							% Left quotation mark, space from right
     \textquotedblright={150, }} 							% Right quotation mark, space from left


\SetExtraKerning[unit=space]
  {encoding={*}, family={qhv}, series={b}, size={large,Large}}
  {1={-200,-200}, 
   \textendash={400,400}}

\SetTracking{encoding={*}, shape=sc}{40} 					% This is for better small caps with tracking and microtype.

\SetProtrusion{encoding={*},family={bch},series={*},size={6,7}}  % This enables better optical views.
              {1={ ,750},2={ ,500},3={ ,500},4={ ,500},5={ ,500},
               6={ ,500},7={ ,600},8={ ,500},9={ ,500},0={ ,500}}

% Page settings
\usepackage{geometry}
 \geometry{
 a4paper,
 %total={210mm,297mm},
 left=15mm,
 right=15mm,
 top=15mm,
 bottom=10mm,
 }

% required when changing page layout lengths
\AtBeginDocument{\recomputelengths} 
\usepackage{xunicode}
\usepackage{xltxtra}
% \usepackage[utf8]{inputenc} (Xelatex expects UTF8 anyway)

% I like pretty logos
\usepackage{metalogo}

% Widen the letter spacing for logos
\setlogokern{La}{0.2pt}
\setlogokern{Xe}{0.2pt}
\setlogokern{eL}{0.2pt}
\setlogokern{Te}{0.2pt}
\setlogokern{aT}{0.2pt}
\setlogokern{eX}{0.2pt}

% Better Quotes (depreciated for this useage)
%\usepackage{epigraph}

% Poaching from Espresso's ug.tex

%%%%%%%%%%%%%%%%%%%%%%%%%%%%%%%%%%%%%%%%%%%%%%%%%%
%%%%%%%%%%%%%%%%%%%%%%%%%%%%%%%%%%%%%%%%%%%%%%%%%%
%%%%%%%%% New Commands and Environments %%%%%%%%%%
%%%%%%%%%%%%%%%%%%%%%%%%%%%%%%%%%%%%%%%%%%%%%%%%%%
%%%%%%%%%%%%%%%%%%%%%%%%%%%%%%%%%%%%%%%%%%%%%%%%%%
\newcommand{\es}{\mbox{\textsf{ESPResSo}}\xspace}

% german word break/hyphenation rules that's with ngerman (we use english)
\usepackage[british,UKenglish,USenglish,american]{babel}

% insert dummy text (used in the letter)
%\usepackage{lipsum} 

% used for \begin{comment}...\end{comment}
\usepackage{verbatim} 

% use guilllemets in bibliography (not german.) Also autostyle superceeded babel
\usepackage[autostyle=true]{csquotes}

\usepackage[
	sorting=none,
	minbibnames=8,
	maxbibnames=9,
	block=space,
  defernumbers=true
]{biblatex}

\bibliography{publications}


% get rid of the number-labels ([1], [2], etc.) in front of publications
\defbibenvironment{midbib}
{\list
	{}
	{
		\setlength{\leftmargin}{0mm}
		\setlength{\itemindent}{-\leftmargin}
		\setlength{\itemsep}{\bibitemsep}
		\setlength{\parsep}{\bibparsep}}
	}
	{\endlist}
{\item}


% add [DOI] and [PDF] fields at the end of each publication entry
\DeclareSourcemap{
	\maps[datatype=bibtex]{
		% the bibtex entry 'mydoi' gets mapped to 'usera'
		\map{
			\step[fieldsource=mydoi]
			\step[fieldset=usera, origfieldval]
		}
		% the bibtex entry 'mypdf' gets mapped to 'usera'
		\map{
			\step[fieldsource=mypdf]
			\step[fieldset=userb, origfieldval]
		}
	}
}

% [DOI] entries in publication
\DeclareFieldFormat{usera}{\color{color1}[\href{#1}{\textsc{doi}}]}
\AtEveryBibitem{
	% put [DOI] stuff at the end of a publication entry
	\csappto{blx@bbx@\thefield{entrytype}}{% 
		\iffieldundef{usera}{ 
			% this gets invoked, once nothing is supplied
			% via the mypdf or mydoi value.
			% you could e.g. display a default thing here.
		}{\space\printfield{usera}}}
}

% [PDF] entries in publication
\DeclareFieldFormat{userb}{\color{color1}[\href{#1}{\textsc{pdf}}]}

\AtEveryBibitem{
	% put [DOI] stuff at the end of a publication entry
	\csappto{blx@bbx@\thefield{entrytype}}{\iffieldundef{userb}{}{\printfield{userb}}}
}

\renewcommand*{\mkbibnamegiven}[1]{%
	\ifitemannotation{highlight}
	{\textbf{#1}}
	{#1}}

	\renewcommand*{\mkbibnamefamily}[1]{%
	\ifitemannotation{highlight}
	{\textbf{#1}}
	{#1}
}


% Minion Pro is used as the main font, if you don't
% have it installed uncomment this line or choose another pretty font.
% Literata is also included.
%\setmainfont[Numbers=OldStyle]{Minion Pro}

\setmainfont[
Path = Fonts/MinionPro/,
Numbers=OldStyle,
Extension = .otf,
UprightFont = *_Regular,
ItalicFont = *_It,
BoldFont = *_Bold,
BoldItalicFont = *_BoldIt
]{MinionPro}

% Myriad Pro is used as the sans font, if you don't
% have it installed uncomment this line or choose another pretty font.
\setsansfont[
Path = Fonts/MyriadPro/,
Numbers=OldStyle,
Extension = .otf,
UprightFont = *_Regular,
ItalicFont = *_It,
BoldFont = *_Bold,
BoldItalicFont = *_BoldIt,
Ligatures=TeX,
Scale=MatchLowercase]{MyriadPro}

% the moderncv package will populate a lot of the pdf meta-information.
% this can be used to change some of these infos.
\AfterPreamble{\hypersetup{
  pdfcreator={XeLaTeX},
  pdftitle={First fledging CV}
}}

% for the icons (telephone, globe). I found the icons provided by the
% fontawesome package prettier than the standard moderncv icons.
\defaultfontfeatures{
    Path = Fonts/FontAwesome/ }
\usepackage{fontawesome}

% personal data
\firstname{Amrita}
\familyname{Goswami}
\address{House No. 646, 35th Lane}{IIT Kanpur, 208016}
\quote{``Avoid the temptation to work so hard that there is no time left for serious thinking.''\\-- Francis Crick}

% \faEnvelope \faPhone \faGithub \faGlobe

% I use the extrainfo command for additional information, since I 
% want to use custom icons and have finer control over spacing.
\extrainfo{
	\faPhone\hspace{0.3em}+91 8874170409\\
	{\small\faEnvelope}\hspace{0.3em}amritag@iitk.ac.in\\
	{\small\faGithub}\hspace{0.3em}amritagos\\
	\faGlobe\hspace{0.3em}https://github.com/amritagos/
}


% picture, resized to a height of 84pt
\photo[84pt]{Picture/amrita}

% spacing before (sub)sections
\newcommand{\spacesection}{\vspace{0.4cm}}
\newcommand{\spacesubsection}{\vspace{0.2cm}}


%===========================
\begin{document}

\maketitle

\section{Personal Data}
\cvitem{Name}{Amrita Goswami}
\cvitem{Date Of Birth}{16.08.1991}
%\cvitem{Geburtsort}{Petrovichi, Russia}

\spacesection
\section{Education}
\cventry{\textsc{2016--present}}{MS-Ph.D. Chemical Engineering}{Indian Institute of Technology}{Kanpur}{\emph{India}}{$8.75$ CGPA (\textsc{Advisor: } Prof. Jayant K. Singh; \textsc{Co-Advisor: } Prof. Indranil Saha Dalal)}
%\cventry{\textsc{1939--1941}}{M.Sc. Chemistry}{University Extension}{New York}{\emph{United States}}{}
\cventry{\textsc{2012--2016}}{B.Tech. Chemical Engineering}{Harcourt Butler Technical University}{Kanpur}{\emph{India}}{$72.36\%$ First Division (\textsc{Project: } Sulphur Acid Production optimization via the Chamber Process)}
\cventry{\textsc{2008--2010}}{Intermediate (AISSCE)}{The Jain International School}{Kanpur}{\emph{India}}{$85\%$ Central Board of Secondary Education (CBSE)}
\cventry{\textsc{2006--2008}}{High School (AISSE)}{Delhi Public School Kalyanpur}{Kanpur}{\emph{India}}{$93\%$ Central Board of Secondary Education (CBSE)}

% Consider downgrading to the bottom
\section{Technical Skills}
\subsection{Programming Languages}
\spacesubsection

\cvdoubleitem{\textsc{Experienced}}{C++(11,17), FORTRAN 90, Tcl, R, C99, Shell (zsh,bash)}
           {\textsc{Familiar}}{Julia, Python(2.7 and 3.6), FORTRAN 2008}

% \newpage
\spacesubsection
\subsection{Simulation Packages}
\spacesubsection

\cvdoubleitem{\textsc{Experienced}}{LAMMPS (Large-scale Atomic /Molecular Massively Parallel Simulator) for Nucleation, Nanoparticles and wetting, VMD (Visual Molecular Dynamics), Ovito}
           {\textsc{Familiar}}{\es \space(Extensible Simulation Package for Research on Soft matter), OpenFOAM, GROMACS (GROningen MAchine for Chemical Simulations), AMBER}

\spacesubsection
\subsection{Tools}
\spacesubsection

\cvdoubleitem{\textsc{Experienced}}{gnuplot,\XeLaTeX, sed, awk, Git (version control), tmux, ssh, Sublime Text Editor 3, Vim, gadfly, i3 (tiling window manager), mosh, babun, MATLAB (matrix laboratory), markdown, Photoshop}
           {\textsc{Familiar}}{moltemplate, Office-Suites (MS, OpenOffice, LibreOffice)}

% \spacesubsection
% \subsection{Operating Systems}
% \cvdoubleitem{\textsc{Preferred}}{ArchLinux}
%            {\textsc{Experienced}}{Windows (95, 2000, XP, 7, 8, 10), MacOS (10.7, 10.11, 10.12), Android (1.5, 1.6, 2.2.*, 2.3.*, 4.0.*, 4.4.*, 5.0.*, 6.0.*, 7.* ), Linux Distros (Ubuntu, Sabyon, Puppy, Manjaro, Debian, Red Hat (CentOS))}
\newpage
% \spacesection
\section{Research Topics}
\cvdoubleitem{\textsc{Experienced}}{Ice nucleation, NEMD, Molecular Dynamics simulations, Phase transitions, Classical Nucleation Theory, Structure elucidation, High performance open source software, Scientific Software Development }
           {\textsc{Interested}}{Molecular modeling, Free energy analysis, Optimal time-stepping methods, Accelerated simulations, HPC Algorithms}

\spacesection
\section{Accolades \& Affiliations}
\subsection{Awards}
\spacesubsection
\cventry{\textsc{November 2019}}{RSC Physical Chemistry Chemical Physics Poster Prize}{DAE Computational Chemistry Symposium, BARC, India}{}{}{}
\cventry{\textsc{February 2020}}{Springer Poster Award}{Molecular Simulations of Complex Fluids and Interfaces, IIT Kanpur}{}{}{}
\cventry{\textsc{November 2019}}{Hot PCCP Article}{}{Article selected as a '2019 HOT PCCP Article', and as an inside front cover}{}{}
% \cventry{\textsc{2014}}{Resonance Short Story Competition}{Harcourt Butler Technological Institute, Kanpur}{First Prize}{}{}
\spacesubsection
\subsection{Memberships}
\spacesubsection
\cventry{\textsc{2014--present}}{AIChE (American Institute Of Chemical Engineers)}{Student Member}{}{}{}
\cventry{\textsc{2018--present}}{OSA (Optical Society of America)}{Student Member}{}{}{}

\spacesection
\section{Experience}
\subsection{Teaching}
\cventry{\textsc{2016--present}}{Teaching Assistant}{Indian Institute of Technology, Kanpur}{}{I have been a teaching assistant for the undergraduate courses 'Chemical Engineering Thermodynamics' and 'ESO-201, Thermodynamics'}{}
\cventry{\textsc{July--August 2020}}{Water, Chemicals and more with Computers for Chemistry (WC3m)}{Wave Learning Festival}{15 hour long summer course for high school students and undergraduates on the basics of computational chemistry}{}{}
\cventry{\textsc{Winter 2020--present}}{NPTEL Chemical Engineering Thermodynamics}{Indian Institute of Technology Kanpur}{}{I am a teaching assistant for an online national course organized by Indian Institutes of Technology and Indian Institute of Science}{}

\subsection{Reviews}

\cventry{\textsc{2019--present}}{Journal of Open Source Software}{Reviewer}{}{}{}

\cventry{\textsc{2019--present}}{PeerJ-Computer Science}{Reviewer}{}{}{}

% PUBLICATIONS
\spacesection
\section{Publications}
 {\color{color1}\textsc{Journals}}

\nocite{*}
% \printbibliography[heading=none, env=midbib, keyword=journal, resetnumbers=true] % no numbering
\printbibliography[heading=none, keyword=journal, resetnumbers=true] % with numbering


% Currently none which haven't been published
{\color{color1}\textsc{Preprints}}

\nocite{*}
\printbibliography[heading=none, env=midbib, keyword=preprint] % without numbering
% \printbibliography[heading=none, keyword=preprint, resetnumbers=false] % with numbering

{\color{color1}\textsc{Conference Proceedings}}

\nocite{*}
% \printbibliography[heading=none, env=midbib, keyword=conference, resetnumbers=true] % without numbering
\printbibliography[heading=none, keyword=conference, resetnumbers=true] % with numbering



% \newpage
\section{Conferences, Symposia \& Workshops}
\spacesection
\subsection{Posters}
\cventry{\textsc{7-9 November 2019}}{DAE Computational Chemistry Symposium, BARC}{Mumbai}{A Family of General Topological Network Criteria for Confined Ice Structure Determination}{}{}
\cventry{\textsc{21-23 February 2020}}{Molecular Simulations of Complex Fluids and Interfaces}{IIT Kanpur}{Formulation and Implementation of General Topological Network Criteria for Exploring the Structures of Confined Ice}{}{}
% \cventry{\textsc{March 2019}}{Space Filling Curves: Heuristics For Semi Classical Lasing Computations}{URSI Asia-Pacific Radio Science Conference (AP-RASC 2019)}{R. Goswami, A. Goswami, and D. Goswami}{}{}
% \cventry{\textsc{December 2018}}{FDTD Numerical Computations for Ultrafast Non-linear Optics}{Photonics-2018}{R. Goswami, A. Goswami, and D. Goswami}{}{}

\spacesection
\subsection{Attended}
\cventry{\textsc{December 2017}}{RARE Symposium}{Agra}{}{}{}
\cventry{\textsc{July 2019}}{Rare Events Summer School}{Indian Institute of Science, Bangalore}{A short course consisting of lectures and hands-on sessions by experts in the field, organized by Prof. Baron Peters}{}{}
\cventry{\textsc{21 September 2019}}{OpenACC GPU Bootcamp}{Indian Institute of Technology, Kanpur}{Day long programming session and discussion covering the acceleration of Institute in-house code by a Senior NVIDIA Solution Architect}{}{}

\section{Relevant Coursework}
\cventry{\textsc{2017 Spring}}{Molecular Modelling In Chemistry}{CHM695}{\textsc{Instructor: } Prof. Nisanth Nair}{\bfseries{Grade: A$^{*}$}}{}
\cventry{\textsc{2017 Fall}}{Intermolecular and Surface Forces}{CHE625A}{\textsc{Instructor: } Prof. Animangsu Ghatak}{\bfseries{Grade: A}}{}
\cventry{\textsc{2016 Spring}}{Introduction to Molecular Simulations}{CHE622A}{\textsc{Instructor: } Prof. Martin Horsch}{\bfseries{Grade: B}}{}

\spacesection
\section{Miscellaneous}
\subsection{Internships}
\cventry{\textsc{Summer 2015}}{Prof. Krishanu Ray}{Tata Institute for Fundamental Research, Mumbai}{}{VSRP Fellow}{Worked on micro-channel flow modeling with OpenFOAM and produced a working prototype with the machine shop of TIFR. Also attended lectures over eight weeks as a part of the program. \\~
 \textsc{Project Report: }Design of a flow-cell for TIRFM imagining of Kinesin-2}
\cventry{\textsc{Winter 2014}}{Prof. Animangsu Ghatak}{Indian Institute of Technology Kanpur}{}{Research Intern}{Worked on the imaging of programmable micro-lenses of oil on a PDMS substrate.}

\subsection{Volunteer Work}

\cventry{\textsc{2017--2018}}{ChemE Research Scholar Day}{Indian Institute of Technology, Kanpur}{}{Anchor}{Managed and spearheaded the festivities of the research oriented student presentations and posters.}

\cventry{\textsc{February 2019}}{FunMolSim Workshop}{Indian Institute of Technology, Kanpur}{Organizer}{Helped organize, designed and taught tutorials at a pedagogical workshop for molecular dynamics}{}

\cventry{\textsc{2017--2020}}{Animal Welfare Group}{Indian Institute of Technology, Kanpur}{}{Member}{Rescued and fostered stray and injured animals.}

% \cventry{\textsc{2012--2016}}{The Curiosity Magazine}{Harcourt Butler Technical University, Kanpur}{}{Editor-in-Chief}{Managed a diverse team of student content writers and rose from the ranks of writer in first year to the Editor-in-Chief by the third year.}

%{\color{color1}\textsc{Fictional Writings}}
%\printbibliography[heading=none, env=midbib, keyword=fiction]



%=============================
% this part is a simple cover letter
%\clearpage

%\recipient{Human Resources}{Some Company Ltd.\\Some Street 123\\12345 Some City} 
%\date{\today}
%\opening{Dear Sir or Madam,}
%\closing{Sincerely yours,}
%\enclosure[Attached]{curriculum vit\ae{}}

%\makelettertitle

%\lipsum[1-3] 

%\makeletterclosing


\end{document}

%%% Local Variables:
%%% mode: latex
%%% TeX-master: t
%%% End:
